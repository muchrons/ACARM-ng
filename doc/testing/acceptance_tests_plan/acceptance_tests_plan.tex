% A4==21x29.7[cm]
\documentclass[a4paper,12pt]{article}
% PL version encoded in UTF-8
%\usepackage{polski}
\usepackage[utf8]{inputenc}
%\usepackage{fancyhdr}
%\usepackage{longtable}

% setup margins
\usepackage{geometry}
%\geometry{verbose,a4paper,tmargin=2cm,bmargin=2cm,lmargin=2cm,rmargin=2cm}

% page numbering off
%\pagestyle{empty}

% make text nicely justified
%\sloppy
% do not leave small pieces on new sides
%\clubpenalty=10000
%\widowpenalty=10000

\usepackage{hyperref}

% use polish indentaion style
\usepackage{indentfirst}
% we'll use graphics
%\usepackage{graphicx}

%\pagestyle{fancy}

%\setlength{\textheight}{20cm}

\title{Acceptance Tests Plan for ACARM-ng project\\(version 0.1.0)}
\author{Bartosz 'BaSz' Szurgot (bartosz.szurgot@pwr.wroc.pl)}

\begin{document}

%
% macros
%

% create new history entry
\newcommand{\historyEntry}[3]
{
#1 & #2 & #3 \\ \hline
}

% create new test case
\newcommand{\testCase}[6]
{
  \subsection{#1: #2}
  \begin{tabular}{ | p{3.5cm} | p{9cm} | }
    \hline
    \textbf{Reference} & #1 \\ \hline
    \textbf{Summary} & #2 \\ \hline
    \textbf{Description} & #3 \\ \hline
    \textbf{Preconditions} & #4 \\ \hline
    \textbf{Plan} & #5 \\ \hline
    \textbf{Post conditions} & #6 \\ \hline
  \end{tabular}
}

%
% document's content
%
\maketitle

\tableofcontents

\section{Introduction}

This document describes testing plan (i.e. test cases to be executed) for the ACARM-ng application
(\url{http://www.acarm.wcss.wroc.pl}). Document has been organized in set of test cases that are to be performed.

\subsection{Test cases}
Each test case is described with the following fields:
\begin{itemize}
  \item reference number -- unique signature assigned to this test -- once assigned it cannot change later on.
  \item summary -- short summary of test (test name).
  \item description -- long description of test case -- what is it testing, what is it needed, etc...
  \item preconditions -- conditions that have to be met before executing test plan.
  \item plan -- steps, point-by-point to be executed in order to perform scan.
  \item post conditions -- final state that should be reached.
\end{itemize}


\subsection{Versioning notes}
When changing this document always change version number and append proper comment into change history.
Version number is organized as follows: X.Y.Z where X is release version indicating big changes in document,
Y is major version number indicating extending document's content (i.e. adding new test cases), Z is minor
version number indicating small fixes in document (typos, updating descriptions of test cases, etc...).



\section{Change history}
\begin{tabular}{ | c | c | l | }
  \hline
  \textbf{Date} & \textbf{Version} & \textbf{Changes} \\ \hline
  % entries goes here - keep them sorted: most recent on the top
  \historyEntry{2010.08.17}{0.1.0}{document creation}
\end{tabular}



\section{Test cases}
\subsection{Functional tests}
\testCase
{f-1}
{IP correlation.}
{Correlating alerts origin from and send to the same host (IP).}
{System is configured wilt all the filters.}
{
\begin{enumerate*}
\item Generate 150 alerts with the same source IP.
\item Send them to the system.
\item Generate 150 alerts with the same target IP.
\item Send them to the system.
\end{enumerate*}
}
{New meta-alert aggregating all the alerts from/to the same host is created for each unique host.}
{Post condition can be easily checked with a trigger or data base content's inspection.}


\testCase
{f-2}
{Reading alerts from prelude-manager.}
{Prelude-manager is one of the possible alert sources. Alerts gathered by it should be accessed by ACARM-ng as well.}
{System is configured with prelude's input.}
{
\begin{enumerate*}
\item Register new sensor for prelude-manager, to be used by ACARM-ng.
\item Configure prelude's input module of ACARM-ng to use newly registered profile.
\item Run ACARM-ng and wait for data from prelude-manager.
\end{enumerate*}
}
{New alerts are gathered by the system.}
{Post condition can be easily checked with a trigger or data base content's inspection.}


\testCase
{f-3}
{Reconnecting to prelude-manager.}
{ACARM-ng must reconnect in case connection to prelude manager is lost.}
{System is configured with prelude's input.}
{
\begin{enumerate*}
\item Run Prelude-Manager.
\item Run ACARM-ng
\item Wait until they are connected.
\item Turn off Prelude-Manager.
\item Wait until ACARM-ng notices dead connection (see logs).
\item Turn on Prelude-Manager.
\end{enumerate*}
}
{ACARM-ng must reconnect to Prelude-Manager.}
{See logs for details on what's going on.}


\testCase
{f-4}
{Connect to PotgreSQL.}
{ACARM-ng supports writing data to PostgreSQL data base.}
{System is configured with postgres persistency module.}
{
\begin{enumerate*}
\item Turn PostgreSQL server on.
\item Run ACARM-ng.
\item Generate some alerts and send it to ACARM-ng.
\end{enumerate*}
}
{New entries should appear in data base.}
{}


\testCase
{f-5}
{Reconnecting to PostgreSQL.}
{ACARM-ng must reconnect in case connection to data base is lost.}
{System is configured with postgres persistency module.}
{
\begin{enumerate*}
\item Run PostgreSQL server.
\item Run ACARM-ng
\item Wait until they are connected.
\item Turn off PostgreSQL.
\item Wait until ACARM-ng notices dead connection (see logs).
\item Turn on PostgreSQL.
\end{enumerate*}
}
{ACARM-ng must reconnect to PostgreSQL.}
{See logs for details on what's going on.}


\testCase
{f-6}
{Trigger reports buffering.}
{When standard trigger failed to send message it is buffered internally and sending is retried next time it is activated. Messages must not be lost, due to temporal failures of external services.}
{System is configured with buffering trigger configured (ex. are: gg, mail).}
{
\begin{enumerate*}
\item Put internet connection up.
\item Run ACARM-ng.
\item item Verify that messages are being sent from system.
\item item Put internet connection down.
\item item Verify that new messages are created and sending failed.
\item item Put internet connection up again.
\end{enumerate*}
}
{Ensure all buffered messages are being sent.}
{See logs for details on what's going on. Notice that message buffer is of a finite length, thus when too many reports are to be sent oldes ones are truncated.}


\subsection{Nonfunctional tests}
\testCase
{nf-1}
{Basic throughput}
{Test maximal number of alerts that can be accepted by the system in a second (assuming that it will be removed straight away).}
{System is up and running, configured without any filters and triggers.}
{
\begin{enumerate}
\item{Start alerts generation}
\item{Measure number of alerts accepted in a given amount of time}
\end{enumerate}
}
{Minimum reasonable number of alerts on a single-core machine is 1000.}


\testCase
{nf-2}
{Proper memory management}
{Check if application does not leak memory when running.}
{System is up and running, configured with filters having minimum allowed time for buffering alerts for correlation.}
{
\begin{enumerate}
\item{Start alerts generation}
\item{Measure number of alerts accepted in a given amount of time}
\end{enumerate}
}
{Minimum reasonable number of alerts on a single-core machine is 1000.}


\end{document}
